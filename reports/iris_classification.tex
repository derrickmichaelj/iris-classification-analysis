% Options for packages loaded elsewhere
% Options for packages loaded elsewhere
\PassOptionsToPackage{unicode}{hyperref}
\PassOptionsToPackage{hyphens}{url}
\PassOptionsToPackage{dvipsnames,svgnames,x11names}{xcolor}
%
\documentclass[
  letterpaper,
  DIV=11,
  numbers=noendperiod]{scrartcl}
\usepackage{xcolor}
\usepackage{amsmath,amssymb}
\setcounter{secnumdepth}{-\maxdimen} % remove section numbering
\usepackage{iftex}
\ifPDFTeX
  \usepackage[T1]{fontenc}
  \usepackage[utf8]{inputenc}
  \usepackage{textcomp} % provide euro and other symbols
\else % if luatex or xetex
  \usepackage{unicode-math} % this also loads fontspec
  \defaultfontfeatures{Scale=MatchLowercase}
  \defaultfontfeatures[\rmfamily]{Ligatures=TeX,Scale=1}
\fi
\usepackage{lmodern}
\ifPDFTeX\else
  % xetex/luatex font selection
\fi
% Use upquote if available, for straight quotes in verbatim environments
\IfFileExists{upquote.sty}{\usepackage{upquote}}{}
\IfFileExists{microtype.sty}{% use microtype if available
  \usepackage[]{microtype}
  \UseMicrotypeSet[protrusion]{basicmath} % disable protrusion for tt fonts
}{}
\makeatletter
\@ifundefined{KOMAClassName}{% if non-KOMA class
  \IfFileExists{parskip.sty}{%
    \usepackage{parskip}
  }{% else
    \setlength{\parindent}{0pt}
    \setlength{\parskip}{6pt plus 2pt minus 1pt}}
}{% if KOMA class
  \KOMAoptions{parskip=half}}
\makeatother
% Make \paragraph and \subparagraph free-standing
\makeatletter
\ifx\paragraph\undefined\else
  \let\oldparagraph\paragraph
  \renewcommand{\paragraph}{
    \@ifstar
      \xxxParagraphStar
      \xxxParagraphNoStar
  }
  \newcommand{\xxxParagraphStar}[1]{\oldparagraph*{#1}\mbox{}}
  \newcommand{\xxxParagraphNoStar}[1]{\oldparagraph{#1}\mbox{}}
\fi
\ifx\subparagraph\undefined\else
  \let\oldsubparagraph\subparagraph
  \renewcommand{\subparagraph}{
    \@ifstar
      \xxxSubParagraphStar
      \xxxSubParagraphNoStar
  }
  \newcommand{\xxxSubParagraphStar}[1]{\oldsubparagraph*{#1}\mbox{}}
  \newcommand{\xxxSubParagraphNoStar}[1]{\oldsubparagraph{#1}\mbox{}}
\fi
\makeatother


\usepackage{longtable,booktabs,array}
\usepackage{calc} % for calculating minipage widths
% Correct order of tables after \paragraph or \subparagraph
\usepackage{etoolbox}
\makeatletter
\patchcmd\longtable{\par}{\if@noskipsec\mbox{}\fi\par}{}{}
\makeatother
% Allow footnotes in longtable head/foot
\IfFileExists{footnotehyper.sty}{\usepackage{footnotehyper}}{\usepackage{footnote}}
\makesavenoteenv{longtable}
\usepackage{graphicx}
\makeatletter
\newsavebox\pandoc@box
\newcommand*\pandocbounded[1]{% scales image to fit in text height/width
  \sbox\pandoc@box{#1}%
  \Gscale@div\@tempa{\textheight}{\dimexpr\ht\pandoc@box+\dp\pandoc@box\relax}%
  \Gscale@div\@tempb{\linewidth}{\wd\pandoc@box}%
  \ifdim\@tempb\p@<\@tempa\p@\let\@tempa\@tempb\fi% select the smaller of both
  \ifdim\@tempa\p@<\p@\scalebox{\@tempa}{\usebox\pandoc@box}%
  \else\usebox{\pandoc@box}%
  \fi%
}
% Set default figure placement to htbp
\def\fps@figure{htbp}
\makeatother


% definitions for citeproc citations
\NewDocumentCommand\citeproctext{}{}
\NewDocumentCommand\citeproc{mm}{%
  \begingroup\def\citeproctext{#2}\cite{#1}\endgroup}
\makeatletter
 % allow citations to break across lines
 \let\@cite@ofmt\@firstofone
 % avoid brackets around text for \cite:
 \def\@biblabel#1{}
 \def\@cite#1#2{{#1\if@tempswa , #2\fi}}
\makeatother
\newlength{\cslhangindent}
\setlength{\cslhangindent}{1.5em}
\newlength{\csllabelwidth}
\setlength{\csllabelwidth}{3em}
\newenvironment{CSLReferences}[2] % #1 hanging-indent, #2 entry-spacing
 {\begin{list}{}{%
  \setlength{\itemindent}{0pt}
  \setlength{\leftmargin}{0pt}
  \setlength{\parsep}{0pt}
  % turn on hanging indent if param 1 is 1
  \ifodd #1
   \setlength{\leftmargin}{\cslhangindent}
   \setlength{\itemindent}{-1\cslhangindent}
  \fi
  % set entry spacing
  \setlength{\itemsep}{#2\baselineskip}}}
 {\end{list}}
\usepackage{calc}
\newcommand{\CSLBlock}[1]{\hfill\break\parbox[t]{\linewidth}{\strut\ignorespaces#1\strut}}
\newcommand{\CSLLeftMargin}[1]{\parbox[t]{\csllabelwidth}{\strut#1\strut}}
\newcommand{\CSLRightInline}[1]{\parbox[t]{\linewidth - \csllabelwidth}{\strut#1\strut}}
\newcommand{\CSLIndent}[1]{\hspace{\cslhangindent}#1}



\setlength{\emergencystretch}{3em} % prevent overfull lines

\providecommand{\tightlist}{%
  \setlength{\itemsep}{0pt}\setlength{\parskip}{0pt}}



 


\KOMAoption{captions}{tableheading}
\makeatletter
\@ifpackageloaded{caption}{}{\usepackage{caption}}
\AtBeginDocument{%
\ifdefined\contentsname
  \renewcommand*\contentsname{Table of contents}
\else
  \newcommand\contentsname{Table of contents}
\fi
\ifdefined\listfigurename
  \renewcommand*\listfigurename{List of Figures}
\else
  \newcommand\listfigurename{List of Figures}
\fi
\ifdefined\listtablename
  \renewcommand*\listtablename{List of Tables}
\else
  \newcommand\listtablename{List of Tables}
\fi
\ifdefined\figurename
  \renewcommand*\figurename{Figure}
\else
  \newcommand\figurename{Figure}
\fi
\ifdefined\tablename
  \renewcommand*\tablename{Table}
\else
  \newcommand\tablename{Table}
\fi
}
\@ifpackageloaded{float}{}{\usepackage{float}}
\floatstyle{ruled}
\@ifundefined{c@chapter}{\newfloat{codelisting}{h}{lop}}{\newfloat{codelisting}{h}{lop}[chapter]}
\floatname{codelisting}{Listing}
\newcommand*\listoflistings{\listof{codelisting}{List of Listings}}
\makeatother
\makeatletter
\makeatother
\makeatletter
\@ifpackageloaded{caption}{}{\usepackage{caption}}
\@ifpackageloaded{subcaption}{}{\usepackage{subcaption}}
\makeatother
\usepackage{bookmark}
\IfFileExists{xurl.sty}{\usepackage{xurl}}{} % add URL line breaks if available
\urlstyle{same}
\hypersetup{
  pdftitle={Infering Relationships Between Iris Species and their Characteristics},
  pdfauthor={Manikanth Goud Gurujala, Aitong Wu, Siharth Malik, Derrick Jaskiel},
  colorlinks=true,
  linkcolor={blue},
  filecolor={Maroon},
  citecolor={Blue},
  urlcolor={Blue},
  pdfcreator={LaTeX via pandoc}}


\title{Infering Relationships Between Iris Species and their
Characteristics}
\author{Manikanth Goud Gurujala, Aitong Wu, Siharth Malik, Derrick
Jaskiel}
\date{}
\begin{document}
\maketitle

\renewcommand*\contentsname{Table of Contents}
{
\hypersetup{linkcolor=}
\setcounter{tocdepth}{3}
\tableofcontents
}

\section{Summary}\label{summary}

The Iris dataset consists of 150 samples of iris flowers, divided evenly
into three species: setosa, versicolor, and virginica (50 samples each).
Each observation contains four continuous morphological measurements:
\texttt{sepal\_length}, \texttt{sepal\_width}, \texttt{petal\_length},
and \texttt{petal\_width}. All four features are recorded in
centimetres. The dataset is well structured, with no missing values, and
includes a balanced class distribution across the three species. A
preliminary summary of the numerical features shows that values vary
substantially between species; for example, setosa flowers tend to have
smaller petal lengths (around 1.4 cm), while virginica flowers exhibit
much larger petal dimensions. These clear differences suggest that the
dataset is suitable for classification tasks, making it an ideal test
case for evaluating machine learning models. In this project, we have
developed a classification model using \textbf{Logistic Regression} to
predict Iris flower species. based on four measurements: sepal length,
sepal width, petal length, and petal width. A baseline
\textbf{DummyClassifier} produced an accuracy of approximately
\texttt{0.33} across cross-validation folds, confirming that the data is
not trivially predictable and that a more sophisticated model is
required. Before training the model, all numerical features were scaled
using StandardScaler, ensuring that differences in measurement units did
not disproportionately influence the classifier. After scaling features
and performing hyperparameter tuning weith randomized search, our final
\textbf{Logistic Regression classifier} achieved strong performance,
with a training accuracy of \texttt{0.983} and a test accuracy of
\texttt{0.90}. The confusion matrix indicates that the most predicitons
were correct, with only a small number of misclassifications occuring
between the \textbf{versicolor} and \textbf{virginica} classes
reflectiong their natural feature similarity. Overall, the model
demostrated high predicive performance on unseen data, though further
refinement or more advnaced model could help reduce raiming
classification overlap.

\section{Introduction}\label{introduction}

A thorough understanding of the geographic distribution and traits of
plants is invaluable to the development of sustainable agriculture and
biodiversity conservation (Joly et al. (2014)). Thus, as the demand for
information and need for biodiversity grow, it becomes even more
important that this task is one that can be performed efficiently and
accurately (Thyagharajan and Kiruba Raji (2018)). One promising method
through which this could be accomplished is machine learning
classification.

The goal of this project is to explore how different physical
measurements of iris flowers relate to species identity and to build a
classification model that can accurately predict species based on these
characteristics. The Iris dataset is a well-known benchmark in machine
learning because it is clean, balanced, and contains clear biological
differences among species. Each flower is described by four continuous
measurements---sepal length, sepal width, petal length, and petal
width---and belongs to one of three species: setosa, versicolor, or
virginica.

Because these measurements reflect real morphological differences, the
dataset provides a natural opportunity to study the relationship between
flower structure and species classification. It also allows us to
evaluate and compare machine learning models in a controlled
environment. In this project, we perform exploratory data analysis to
understand feature distributions and correlations, then develop a
predictive model using Logistic Regression. We also include a
DummyClassifier as a baseline to ensure that model performance
improvements are meaningful rather than accidental.

Finally, we evaluate the model using cross-validation, hyperparameter
tuning, and confusion-matrix diagnostics. This approach allows us to
understand both how well the model learns from the training data and how
reliably it generalizes to unseen samples. Overall, this project
demonstrates the connection between biological measurements and species
identity while providing a clear example of building, tuning, and
interpreting a supervised classification model.

\section{Methods}\label{methods}

\subsection{Data}\label{data}

The dataset used in this project is the classic Iris flower dataset
originally collected by the British statistician and biologist Ronald A.
Fisher in 1936 as part of his research on linear discriminant analysis.
It is publicly available through the UCI Machine Learning Repository,
where it is widely used as a benchmark dataset for classification tasks.
Each row in the dataset represents physical measurements of a single
iris flower, including four numerical attributes sepal length, sepal
width, petal length, and petal width recorded in centimetres. Alongside
these measurements, each observation is labelled with one of three
species (Iris setosa, Iris versicolor, or Iris virginica), originally
identified by botanists through morphological characteristics. This
dataset is clean, balanced, and well-suited for evaluating machine
learning classification algorithms.

\subsection{Analysis}\label{analysis}

In this step, we load the Iris dataset from an online source into a
pandas DataFrame. We check for missing values to ensure the dataset is
complete, view the first and last few rows to get a sense of the data
structure, and look at the shape and data types to understand what kind
of data we are working with.

\begin{verbatim}

=== RUNNING IRIS DATA VALIDATION CHECKS ===
Data loaded as pandas DataFrame.
Duplicate check passed (duplicates removed if present).
Schema validation passed.
Outlier check passed.
Species category level check passed (unknown categories removed if present).
Target Distribution check passed
Correlation check (features vs target) passed.
Correlation check (feature vs feature) passed.
=== IRIS DATA VALIDATION — ALL CHECKS PASSED ===
\end{verbatim}

\begin{verbatim}
sepal_length    0
sepal_width     0
petal_length    0
petal_width     0
species         0
dtype: int64
\end{verbatim}

\begin{longtable}[]{@{}llllll@{}}
\toprule\noalign{}
& sepal\_length & sepal\_width & petal\_length & petal\_width &
species \\
\midrule\noalign{}
\endhead
\bottomrule\noalign{}
\endlastfoot
0 & 5.1 & 3.5 & 1.4 & 0.2 & setosa \\
1 & 4.9 & 3.0 & 1.4 & 0.2 & setosa \\
2 & 4.7 & 3.2 & 1.3 & 0.2 & setosa \\
3 & 4.6 & 3.1 & 1.5 & 0.2 & setosa \\
4 & 5.0 & 3.6 & 1.4 & 0.2 & setosa \\
\end{longtable}

\begin{longtable}[]{@{}llllll@{}}
\toprule\noalign{}
& sepal\_length & sepal\_width & petal\_length & petal\_width &
species \\
\midrule\noalign{}
\endhead
\bottomrule\noalign{}
\endlastfoot
145 & 6.7 & 3.0 & 5.2 & 2.3 & virginica \\
146 & 6.3 & 2.5 & 5.0 & 1.9 & virginica \\
147 & 6.5 & 3.0 & 5.2 & 2.0 & virginica \\
148 & 6.2 & 3.4 & 5.4 & 2.3 & virginica \\
149 & 5.9 & 3.0 & 5.1 & 1.8 & virginica \\
\end{longtable}

\begin{verbatim}
(149, 5)
\end{verbatim}

\begin{verbatim}
sepal_length    float64
sepal_width     float64
petal_length    float64
petal_width     float64
species          object
dtype: object
\end{verbatim}

\begin{longtable}[]{@{}lllll@{}}
\toprule\noalign{}
& sepal\_length & sepal\_width & petal\_length & petal\_width \\
\midrule\noalign{}
\endhead
\bottomrule\noalign{}
\endlastfoot
count & 149.000000 & 149.000000 & 149.000000 & 149.000000 \\
mean & 5.843624 & 3.059732 & 3.748993 & 1.194631 \\
std & 0.830851 & 0.436342 & 1.767791 & 0.762622 \\
min & 4.300000 & 2.000000 & 1.000000 & 0.100000 \\
25\% & 5.100000 & 2.800000 & 1.600000 & 0.300000 \\
50\% & 5.800000 & 3.000000 & 4.300000 & 1.300000 \\
75\% & 6.400000 & 3.300000 & 5.100000 & 1.800000 \\
max & 7.900000 & 4.400000 & 6.900000 & 2.500000 \\
\end{longtable}

\subsubsection{Data validation checks}\label{data-validation-checks}

\begin{itemize}
\item
  Correct data file format
\item
  Correct column names as per schema
\item
  No empty observations
\item
  Missingness within expected thresholds
\item
  Correct data types for each column
\item
  No duplicate observations
\item
  Values within expected ranges / valid categories
\end{itemize}

\subsubsection{Insights from Data}\label{insights-from-data}

From the above code cells, we can see:

\begin{itemize}
\item
  The dataset has 150 rows and 5 columns.
\item
  There are no missing values, so the data is complete.
\item
  The dataset contains three species: Setosa, Versicolor, and Virginica.
\item
  Petal measurements show the strongest separation between species.
\item
  Setosa is clearly distinct, while Versicolor and Virginica overlap
  somewhat.
\item
  Sepal measurements show smaller differences and are less useful for
  distinguishing species.
\end{itemize}

\subsubsection{Scatter plot}\label{scatter-plot}

This plot shows how the species are separated based on petal
measurements. The three species form distinct clusters, indicating that
these two features are good for classification. It also shows that
Setosa is well-separated, while Versicolor and Virginica have some
overlap.

\begin{figure}

\centering{

\includegraphics[width=0.8\linewidth,height=\textheight,keepaspectratio]{../results/figures/scatter_petal.png}

}

\caption{\label{fig-scatter_petal}}

\end{figure}%

\subsubsection{Boxplots}\label{boxplots}

These plots show the distribution of each feature for different species.
By comparing medians and ranges, we see that Setosa generally has
smaller petals and sepals, while Virginica has the largest. This helps
us understand the differences between species and why some features are
better for classification.

\begin{figure}

\centering{

\includegraphics[width=0.8\linewidth,height=\textheight,keepaspectratio]{../results/figures/boxplots.png}

}

\caption{\label{fig-scatter_petal}}

\end{figure}%

\subsubsection{Correlation Heatmap}\label{correlation-heatmap}

This plot shows how the numeric features are related to each other. We
can see that petal length and petal width are highly correlated, while
sepal length and sepal width have a weaker correlation. This suggests
that petal measurements might be more useful for distinguishing species.

\begin{figure}

\centering{

\includegraphics[width=0.8\linewidth,height=\textheight,keepaspectratio]{../results/figures/correlation_heatmap.png}

}

\caption{\label{fig-scatter_petal}}

\end{figure}%

\section{Results \& Discussion}\label{results-discussion}

In our EDA, we observed that though the pedal attributes (length and
width) show a higher level of interspecies distinction than the sepal
attributes, all features exhibit an adequate amount of spread to be
considered useful for our model. Additionally, our correlation matrix
shows that sepal attributes are indeed correlated with petal attributes,
increasing their utility to our model. After finding no conclusive
evidence in support of dropping features from our model, we move forward
with the full set of features from the dataset.

The next step of the process consists of splitting the data into train
and test. In this case a 80-20\% split is being considered.

As a precautinary step, it is always beneficial to run the dummy model
on the data to get the baseline accuracy. This aids in tuning the true
regression model being used for training and prediction so that a
balanced train-test score can be achieved.

\begin{longtable}[]{@{}lllll@{}}
\toprule\noalign{}
& fit\_time & score\_time & test\_score & train\_score \\
\midrule\noalign{}
\endhead
\bottomrule\noalign{}
\endlastfoot
0 & 0.001489 & 0.001073 & 0.333333 & 0.336842 \\
1 & 0.001251 & 0.000986 & 0.333333 & 0.347368 \\
2 & 0.001549 & 0.000870 & 0.333333 & 0.347368 \\
3 & 0.001386 & 0.000847 & 0.333333 & 0.347368 \\
4 & 0.001434 & 0.000871 & 0.347826 & 0.343750 \\
\end{longtable}

In order to insert all the columns that require column transformations
such as the StandardScaler(), we need to obtain all the feature columns
from the dataset

\begin{verbatim}
['sepal_length', 'sepal_width', 'petal_length', 'petal_width']
\end{verbatim}

The preprocessor is required and is a good practice before using the
pipeline. The preprocessor consists of all the required transformations
and the features on which they will be performed. We employ a
classification model utilizing Logistic Regresison for this task, as we
have a relatively small number of features in our model. This will also
allow us to retain a higher level of interpretability in our model.

To optimize the model's classification capabilities, we employ a
randomized hyperparameter search. This will provide insight into what
values of the C hyperparameter from logistic regression will perform
best in our model.

\begin{longtable}[]{@{}llll@{}}
\toprule\noalign{}
& param\_classifier\_\_C & mean\_test\_score & std\_test\_score \\
\midrule\noalign{}
\endhead
\bottomrule\noalign{}
\endlastfoot
15 & 14.062606 & 0.983333 & 0.020412 \\
1 & 78253.381859 & 0.975000 & 0.020412 \\
17 & 8108.175729 & 0.975000 & 0.020412 \\
3 & 1.712861 & 0.975000 & 0.033333 \\
7 & 620.951562 & 0.975000 & 0.020412 \\
5 & 1.905181 & 0.975000 & 0.033333 \\
9 & 2.054645 & 0.975000 & 0.033333 \\
16 & 3.753517 & 0.975000 & 0.033333 \\
13 & 3.621417 & 0.975000 & 0.033333 \\
6 & 301.128937 & 0.975000 & 0.020412 \\
\end{longtable}

We see that most of our highest-performing models have a C
hyperparameter value in the rough range of 1.75 to 3.75, signalling that
this is our optimal range. Next, we will test our optimized model on our
testing data from our initial train/test data split, allowing us to
observe how well our model generalizes to unseen data.

\begin{verbatim}
Train accuracy: 0.9831932773109243
Test accuracy: 0.9333333333333333
\end{verbatim}

For greater context around these scores, we will plot a confusion
matrix.

\begin{figure}

\centering{

\includegraphics[width=0.8\linewidth,height=\textheight,keepaspectratio]{../results/metrics/confusion_matrix.png}

}

\caption{\label{fig-scatter_petal}}

\end{figure}%

Achieving a training score of 0.98, it is evident that the model was
able to learn the relationships between the four iris measurements.
Additionally, the model generalizes to the unseen data quite well,
yielding a test score of 0.9. While the discrepancy between the model's
train and test scores elicit some concern around potential overfitting
in the model, it is not large enough to overshadow its strong
generalization capabilities.

These findings imply that the four measured flower characteristics are
strong predictors that could reliably enhance species identification.
Future questions could explore whether more complex models could grow
accuracy even further or whether certain species pairs remain more
difficult to separate due to particular traits. Overall, it appears that
this model is well on its way to becoming a strong tool for iris
identification.

\phantomsection\label{refs}
\begin{CSLReferences}{1}{0}
\bibitem[\citeproctext]{ref-Joly_et_al_2014}
Joly, Alexis, Hervé Goëau, Pierre Bonnet, Vera Bakić, Julien Barbe,
Souheil Selmi, Itheri Yahiaoui, Jennifer Carré, Elise Mouysset, and
Jean-François et al. Molino. 2014. {``Interactive Plant Identification
Based on Social Image Data.''} \emph{Ecological Informatics} 23
(September): 22--34. \url{https://doi.org/10.1016/j.ecoinf.2013.07.006}.

\bibitem[\citeproctext]{ref-Thyagharajan_Kiruba_2018}
Thyagharajan, K. K., and I. Kiruba Raji. 2018. {``A Review of Visual
Descriptors and Classification Techniques Used in Leaf Species
Identification.''} \emph{Archives of Computational Methods in
Engineering} 26 (4): 933--60.
\url{https://doi.org/10.1007/s11831-018-9266-3}.

\end{CSLReferences}




\end{document}
